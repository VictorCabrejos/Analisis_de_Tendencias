
% Default to the notebook output style

    


% Inherit from the specified cell style.




    
\documentclass[11pt]{article}

    
    
    \usepackage[T1]{fontenc}
    % Nicer default font (+ math font) than Computer Modern for most use cases
    \usepackage{mathpazo}

    % Basic figure setup, for now with no caption control since it's done
    % automatically by Pandoc (which extracts ![](path) syntax from Markdown).
    \usepackage{graphicx}
    % We will generate all images so they have a width \maxwidth. This means
    % that they will get their normal width if they fit onto the page, but
    % are scaled down if they would overflow the margins.
    \makeatletter
    \def\maxwidth{\ifdim\Gin@nat@width>\linewidth\linewidth
    \else\Gin@nat@width\fi}
    \makeatother
    \let\Oldincludegraphics\includegraphics
    % Set max figure width to be 80% of text width, for now hardcoded.
    \renewcommand{\includegraphics}[1]{\Oldincludegraphics[width=.8\maxwidth]{#1}}
    % Ensure that by default, figures have no caption (until we provide a
    % proper Figure object with a Caption API and a way to capture that
    % in the conversion process - todo).
    \usepackage{caption}
    \DeclareCaptionLabelFormat{nolabel}{}
    \captionsetup{labelformat=nolabel}

    \usepackage{adjustbox} % Used to constrain images to a maximum size 
    \usepackage{xcolor} % Allow colors to be defined
    \usepackage{enumerate} % Needed for markdown enumerations to work
    \usepackage{geometry} % Used to adjust the document margins
    \usepackage{amsmath} % Equations
    \usepackage{amssymb} % Equations
    \usepackage{textcomp} % defines textquotesingle
    % Hack from http://tex.stackexchange.com/a/47451/13684:
    \AtBeginDocument{%
        \def\PYZsq{\textquotesingle}% Upright quotes in Pygmentized code
    }
    \usepackage{upquote} % Upright quotes for verbatim code
    \usepackage{eurosym} % defines \euro
    \usepackage[mathletters]{ucs} % Extended unicode (utf-8) support
    \usepackage[utf8x]{inputenc} % Allow utf-8 characters in the tex document
    \usepackage{fancyvrb} % verbatim replacement that allows latex
    \usepackage{grffile} % extends the file name processing of package graphics 
                         % to support a larger range 
    % The hyperref package gives us a pdf with properly built
    % internal navigation ('pdf bookmarks' for the table of contents,
    % internal cross-reference links, web links for URLs, etc.)
    \usepackage{hyperref}
    \usepackage{longtable} % longtable support required by pandoc >1.10
    \usepackage{booktabs}  % table support for pandoc > 1.12.2
    \usepackage[inline]{enumitem} % IRkernel/repr support (it uses the enumerate* environment)
    \usepackage[normalem]{ulem} % ulem is needed to support strikethroughs (\sout)
                                % normalem makes italics be italics, not underlines
    

    
    
    % Colors for the hyperref package
    \definecolor{urlcolor}{rgb}{0,.145,.698}
    \definecolor{linkcolor}{rgb}{.71,0.21,0.01}
    \definecolor{citecolor}{rgb}{.12,.54,.11}

    % ANSI colors
    \definecolor{ansi-black}{HTML}{3E424D}
    \definecolor{ansi-black-intense}{HTML}{282C36}
    \definecolor{ansi-red}{HTML}{E75C58}
    \definecolor{ansi-red-intense}{HTML}{B22B31}
    \definecolor{ansi-green}{HTML}{00A250}
    \definecolor{ansi-green-intense}{HTML}{007427}
    \definecolor{ansi-yellow}{HTML}{DDB62B}
    \definecolor{ansi-yellow-intense}{HTML}{B27D12}
    \definecolor{ansi-blue}{HTML}{208FFB}
    \definecolor{ansi-blue-intense}{HTML}{0065CA}
    \definecolor{ansi-magenta}{HTML}{D160C4}
    \definecolor{ansi-magenta-intense}{HTML}{A03196}
    \definecolor{ansi-cyan}{HTML}{60C6C8}
    \definecolor{ansi-cyan-intense}{HTML}{258F8F}
    \definecolor{ansi-white}{HTML}{C5C1B4}
    \definecolor{ansi-white-intense}{HTML}{A1A6B2}

    % commands and environments needed by pandoc snippets
    % extracted from the output of `pandoc -s`
    \providecommand{\tightlist}{%
      \setlength{\itemsep}{0pt}\setlength{\parskip}{0pt}}
    \DefineVerbatimEnvironment{Highlighting}{Verbatim}{commandchars=\\\{\}}
    % Add ',fontsize=\small' for more characters per line
    \newenvironment{Shaded}{}{}
    \newcommand{\KeywordTok}[1]{\textcolor[rgb]{0.00,0.44,0.13}{\textbf{{#1}}}}
    \newcommand{\DataTypeTok}[1]{\textcolor[rgb]{0.56,0.13,0.00}{{#1}}}
    \newcommand{\DecValTok}[1]{\textcolor[rgb]{0.25,0.63,0.44}{{#1}}}
    \newcommand{\BaseNTok}[1]{\textcolor[rgb]{0.25,0.63,0.44}{{#1}}}
    \newcommand{\FloatTok}[1]{\textcolor[rgb]{0.25,0.63,0.44}{{#1}}}
    \newcommand{\CharTok}[1]{\textcolor[rgb]{0.25,0.44,0.63}{{#1}}}
    \newcommand{\StringTok}[1]{\textcolor[rgb]{0.25,0.44,0.63}{{#1}}}
    \newcommand{\CommentTok}[1]{\textcolor[rgb]{0.38,0.63,0.69}{\textit{{#1}}}}
    \newcommand{\OtherTok}[1]{\textcolor[rgb]{0.00,0.44,0.13}{{#1}}}
    \newcommand{\AlertTok}[1]{\textcolor[rgb]{1.00,0.00,0.00}{\textbf{{#1}}}}
    \newcommand{\FunctionTok}[1]{\textcolor[rgb]{0.02,0.16,0.49}{{#1}}}
    \newcommand{\RegionMarkerTok}[1]{{#1}}
    \newcommand{\ErrorTok}[1]{\textcolor[rgb]{1.00,0.00,0.00}{\textbf{{#1}}}}
    \newcommand{\NormalTok}[1]{{#1}}
    
    % Additional commands for more recent versions of Pandoc
    \newcommand{\ConstantTok}[1]{\textcolor[rgb]{0.53,0.00,0.00}{{#1}}}
    \newcommand{\SpecialCharTok}[1]{\textcolor[rgb]{0.25,0.44,0.63}{{#1}}}
    \newcommand{\VerbatimStringTok}[1]{\textcolor[rgb]{0.25,0.44,0.63}{{#1}}}
    \newcommand{\SpecialStringTok}[1]{\textcolor[rgb]{0.73,0.40,0.53}{{#1}}}
    \newcommand{\ImportTok}[1]{{#1}}
    \newcommand{\DocumentationTok}[1]{\textcolor[rgb]{0.73,0.13,0.13}{\textit{{#1}}}}
    \newcommand{\AnnotationTok}[1]{\textcolor[rgb]{0.38,0.63,0.69}{\textbf{\textit{{#1}}}}}
    \newcommand{\CommentVarTok}[1]{\textcolor[rgb]{0.38,0.63,0.69}{\textbf{\textit{{#1}}}}}
    \newcommand{\VariableTok}[1]{\textcolor[rgb]{0.10,0.09,0.49}{{#1}}}
    \newcommand{\ControlFlowTok}[1]{\textcolor[rgb]{0.00,0.44,0.13}{\textbf{{#1}}}}
    \newcommand{\OperatorTok}[1]{\textcolor[rgb]{0.40,0.40,0.40}{{#1}}}
    \newcommand{\BuiltInTok}[1]{{#1}}
    \newcommand{\ExtensionTok}[1]{{#1}}
    \newcommand{\PreprocessorTok}[1]{\textcolor[rgb]{0.74,0.48,0.00}{{#1}}}
    \newcommand{\AttributeTok}[1]{\textcolor[rgb]{0.49,0.56,0.16}{{#1}}}
    \newcommand{\InformationTok}[1]{\textcolor[rgb]{0.38,0.63,0.69}{\textbf{\textit{{#1}}}}}
    \newcommand{\WarningTok}[1]{\textcolor[rgb]{0.38,0.63,0.69}{\textbf{\textit{{#1}}}}}
    
    
    % Define a nice break command that doesn't care if a line doesn't already
    % exist.
    \def\br{\hspace*{\fill} \\* }
    % Math Jax compatability definitions
    \def\gt{>}
    \def\lt{<}
    % Document parameters
    \title{Resolucion De A?o Nuevo}
    
    
    

    % Pygments definitions
    
\makeatletter
\def\PY@reset{\let\PY@it=\relax \let\PY@bf=\relax%
    \let\PY@ul=\relax \let\PY@tc=\relax%
    \let\PY@bc=\relax \let\PY@ff=\relax}
\def\PY@tok#1{\csname PY@tok@#1\endcsname}
\def\PY@toks#1+{\ifx\relax#1\empty\else%
    \PY@tok{#1}\expandafter\PY@toks\fi}
\def\PY@do#1{\PY@bc{\PY@tc{\PY@ul{%
    \PY@it{\PY@bf{\PY@ff{#1}}}}}}}
\def\PY#1#2{\PY@reset\PY@toks#1+\relax+\PY@do{#2}}

\expandafter\def\csname PY@tok@w\endcsname{\def\PY@tc##1{\textcolor[rgb]{0.73,0.73,0.73}{##1}}}
\expandafter\def\csname PY@tok@c\endcsname{\let\PY@it=\textit\def\PY@tc##1{\textcolor[rgb]{0.25,0.50,0.50}{##1}}}
\expandafter\def\csname PY@tok@cp\endcsname{\def\PY@tc##1{\textcolor[rgb]{0.74,0.48,0.00}{##1}}}
\expandafter\def\csname PY@tok@k\endcsname{\let\PY@bf=\textbf\def\PY@tc##1{\textcolor[rgb]{0.00,0.50,0.00}{##1}}}
\expandafter\def\csname PY@tok@kp\endcsname{\def\PY@tc##1{\textcolor[rgb]{0.00,0.50,0.00}{##1}}}
\expandafter\def\csname PY@tok@kt\endcsname{\def\PY@tc##1{\textcolor[rgb]{0.69,0.00,0.25}{##1}}}
\expandafter\def\csname PY@tok@o\endcsname{\def\PY@tc##1{\textcolor[rgb]{0.40,0.40,0.40}{##1}}}
\expandafter\def\csname PY@tok@ow\endcsname{\let\PY@bf=\textbf\def\PY@tc##1{\textcolor[rgb]{0.67,0.13,1.00}{##1}}}
\expandafter\def\csname PY@tok@nb\endcsname{\def\PY@tc##1{\textcolor[rgb]{0.00,0.50,0.00}{##1}}}
\expandafter\def\csname PY@tok@nf\endcsname{\def\PY@tc##1{\textcolor[rgb]{0.00,0.00,1.00}{##1}}}
\expandafter\def\csname PY@tok@nc\endcsname{\let\PY@bf=\textbf\def\PY@tc##1{\textcolor[rgb]{0.00,0.00,1.00}{##1}}}
\expandafter\def\csname PY@tok@nn\endcsname{\let\PY@bf=\textbf\def\PY@tc##1{\textcolor[rgb]{0.00,0.00,1.00}{##1}}}
\expandafter\def\csname PY@tok@ne\endcsname{\let\PY@bf=\textbf\def\PY@tc##1{\textcolor[rgb]{0.82,0.25,0.23}{##1}}}
\expandafter\def\csname PY@tok@nv\endcsname{\def\PY@tc##1{\textcolor[rgb]{0.10,0.09,0.49}{##1}}}
\expandafter\def\csname PY@tok@no\endcsname{\def\PY@tc##1{\textcolor[rgb]{0.53,0.00,0.00}{##1}}}
\expandafter\def\csname PY@tok@nl\endcsname{\def\PY@tc##1{\textcolor[rgb]{0.63,0.63,0.00}{##1}}}
\expandafter\def\csname PY@tok@ni\endcsname{\let\PY@bf=\textbf\def\PY@tc##1{\textcolor[rgb]{0.60,0.60,0.60}{##1}}}
\expandafter\def\csname PY@tok@na\endcsname{\def\PY@tc##1{\textcolor[rgb]{0.49,0.56,0.16}{##1}}}
\expandafter\def\csname PY@tok@nt\endcsname{\let\PY@bf=\textbf\def\PY@tc##1{\textcolor[rgb]{0.00,0.50,0.00}{##1}}}
\expandafter\def\csname PY@tok@nd\endcsname{\def\PY@tc##1{\textcolor[rgb]{0.67,0.13,1.00}{##1}}}
\expandafter\def\csname PY@tok@s\endcsname{\def\PY@tc##1{\textcolor[rgb]{0.73,0.13,0.13}{##1}}}
\expandafter\def\csname PY@tok@sd\endcsname{\let\PY@it=\textit\def\PY@tc##1{\textcolor[rgb]{0.73,0.13,0.13}{##1}}}
\expandafter\def\csname PY@tok@si\endcsname{\let\PY@bf=\textbf\def\PY@tc##1{\textcolor[rgb]{0.73,0.40,0.53}{##1}}}
\expandafter\def\csname PY@tok@se\endcsname{\let\PY@bf=\textbf\def\PY@tc##1{\textcolor[rgb]{0.73,0.40,0.13}{##1}}}
\expandafter\def\csname PY@tok@sr\endcsname{\def\PY@tc##1{\textcolor[rgb]{0.73,0.40,0.53}{##1}}}
\expandafter\def\csname PY@tok@ss\endcsname{\def\PY@tc##1{\textcolor[rgb]{0.10,0.09,0.49}{##1}}}
\expandafter\def\csname PY@tok@sx\endcsname{\def\PY@tc##1{\textcolor[rgb]{0.00,0.50,0.00}{##1}}}
\expandafter\def\csname PY@tok@m\endcsname{\def\PY@tc##1{\textcolor[rgb]{0.40,0.40,0.40}{##1}}}
\expandafter\def\csname PY@tok@gh\endcsname{\let\PY@bf=\textbf\def\PY@tc##1{\textcolor[rgb]{0.00,0.00,0.50}{##1}}}
\expandafter\def\csname PY@tok@gu\endcsname{\let\PY@bf=\textbf\def\PY@tc##1{\textcolor[rgb]{0.50,0.00,0.50}{##1}}}
\expandafter\def\csname PY@tok@gd\endcsname{\def\PY@tc##1{\textcolor[rgb]{0.63,0.00,0.00}{##1}}}
\expandafter\def\csname PY@tok@gi\endcsname{\def\PY@tc##1{\textcolor[rgb]{0.00,0.63,0.00}{##1}}}
\expandafter\def\csname PY@tok@gr\endcsname{\def\PY@tc##1{\textcolor[rgb]{1.00,0.00,0.00}{##1}}}
\expandafter\def\csname PY@tok@ge\endcsname{\let\PY@it=\textit}
\expandafter\def\csname PY@tok@gs\endcsname{\let\PY@bf=\textbf}
\expandafter\def\csname PY@tok@gp\endcsname{\let\PY@bf=\textbf\def\PY@tc##1{\textcolor[rgb]{0.00,0.00,0.50}{##1}}}
\expandafter\def\csname PY@tok@go\endcsname{\def\PY@tc##1{\textcolor[rgb]{0.53,0.53,0.53}{##1}}}
\expandafter\def\csname PY@tok@gt\endcsname{\def\PY@tc##1{\textcolor[rgb]{0.00,0.27,0.87}{##1}}}
\expandafter\def\csname PY@tok@err\endcsname{\def\PY@bc##1{\setlength{\fboxsep}{0pt}\fcolorbox[rgb]{1.00,0.00,0.00}{1,1,1}{\strut ##1}}}
\expandafter\def\csname PY@tok@kc\endcsname{\let\PY@bf=\textbf\def\PY@tc##1{\textcolor[rgb]{0.00,0.50,0.00}{##1}}}
\expandafter\def\csname PY@tok@kd\endcsname{\let\PY@bf=\textbf\def\PY@tc##1{\textcolor[rgb]{0.00,0.50,0.00}{##1}}}
\expandafter\def\csname PY@tok@kn\endcsname{\let\PY@bf=\textbf\def\PY@tc##1{\textcolor[rgb]{0.00,0.50,0.00}{##1}}}
\expandafter\def\csname PY@tok@kr\endcsname{\let\PY@bf=\textbf\def\PY@tc##1{\textcolor[rgb]{0.00,0.50,0.00}{##1}}}
\expandafter\def\csname PY@tok@bp\endcsname{\def\PY@tc##1{\textcolor[rgb]{0.00,0.50,0.00}{##1}}}
\expandafter\def\csname PY@tok@fm\endcsname{\def\PY@tc##1{\textcolor[rgb]{0.00,0.00,1.00}{##1}}}
\expandafter\def\csname PY@tok@vc\endcsname{\def\PY@tc##1{\textcolor[rgb]{0.10,0.09,0.49}{##1}}}
\expandafter\def\csname PY@tok@vg\endcsname{\def\PY@tc##1{\textcolor[rgb]{0.10,0.09,0.49}{##1}}}
\expandafter\def\csname PY@tok@vi\endcsname{\def\PY@tc##1{\textcolor[rgb]{0.10,0.09,0.49}{##1}}}
\expandafter\def\csname PY@tok@vm\endcsname{\def\PY@tc##1{\textcolor[rgb]{0.10,0.09,0.49}{##1}}}
\expandafter\def\csname PY@tok@sa\endcsname{\def\PY@tc##1{\textcolor[rgb]{0.73,0.13,0.13}{##1}}}
\expandafter\def\csname PY@tok@sb\endcsname{\def\PY@tc##1{\textcolor[rgb]{0.73,0.13,0.13}{##1}}}
\expandafter\def\csname PY@tok@sc\endcsname{\def\PY@tc##1{\textcolor[rgb]{0.73,0.13,0.13}{##1}}}
\expandafter\def\csname PY@tok@dl\endcsname{\def\PY@tc##1{\textcolor[rgb]{0.73,0.13,0.13}{##1}}}
\expandafter\def\csname PY@tok@s2\endcsname{\def\PY@tc##1{\textcolor[rgb]{0.73,0.13,0.13}{##1}}}
\expandafter\def\csname PY@tok@sh\endcsname{\def\PY@tc##1{\textcolor[rgb]{0.73,0.13,0.13}{##1}}}
\expandafter\def\csname PY@tok@s1\endcsname{\def\PY@tc##1{\textcolor[rgb]{0.73,0.13,0.13}{##1}}}
\expandafter\def\csname PY@tok@mb\endcsname{\def\PY@tc##1{\textcolor[rgb]{0.40,0.40,0.40}{##1}}}
\expandafter\def\csname PY@tok@mf\endcsname{\def\PY@tc##1{\textcolor[rgb]{0.40,0.40,0.40}{##1}}}
\expandafter\def\csname PY@tok@mh\endcsname{\def\PY@tc##1{\textcolor[rgb]{0.40,0.40,0.40}{##1}}}
\expandafter\def\csname PY@tok@mi\endcsname{\def\PY@tc##1{\textcolor[rgb]{0.40,0.40,0.40}{##1}}}
\expandafter\def\csname PY@tok@il\endcsname{\def\PY@tc##1{\textcolor[rgb]{0.40,0.40,0.40}{##1}}}
\expandafter\def\csname PY@tok@mo\endcsname{\def\PY@tc##1{\textcolor[rgb]{0.40,0.40,0.40}{##1}}}
\expandafter\def\csname PY@tok@ch\endcsname{\let\PY@it=\textit\def\PY@tc##1{\textcolor[rgb]{0.25,0.50,0.50}{##1}}}
\expandafter\def\csname PY@tok@cm\endcsname{\let\PY@it=\textit\def\PY@tc##1{\textcolor[rgb]{0.25,0.50,0.50}{##1}}}
\expandafter\def\csname PY@tok@cpf\endcsname{\let\PY@it=\textit\def\PY@tc##1{\textcolor[rgb]{0.25,0.50,0.50}{##1}}}
\expandafter\def\csname PY@tok@c1\endcsname{\let\PY@it=\textit\def\PY@tc##1{\textcolor[rgb]{0.25,0.50,0.50}{##1}}}
\expandafter\def\csname PY@tok@cs\endcsname{\let\PY@it=\textit\def\PY@tc##1{\textcolor[rgb]{0.25,0.50,0.50}{##1}}}

\def\PYZbs{\char`\\}
\def\PYZus{\char`\_}
\def\PYZob{\char`\{}
\def\PYZcb{\char`\}}
\def\PYZca{\char`\^}
\def\PYZam{\char`\&}
\def\PYZlt{\char`\<}
\def\PYZgt{\char`\>}
\def\PYZsh{\char`\#}
\def\PYZpc{\char`\%}
\def\PYZdl{\char`\$}
\def\PYZhy{\char`\-}
\def\PYZsq{\char`\'}
\def\PYZdq{\char`\"}
\def\PYZti{\char`\~}
% for compatibility with earlier versions
\def\PYZat{@}
\def\PYZlb{[}
\def\PYZrb{]}
\makeatother


    % Exact colors from NB
    \definecolor{incolor}{rgb}{0.0, 0.0, 0.5}
    \definecolor{outcolor}{rgb}{0.545, 0.0, 0.0}



    
    % Prevent overflowing lines due to hard-to-break entities
    \sloppy 
    % Setup hyperref package
    \hypersetup{
      breaklinks=true,  % so long urls are correctly broken across lines
      colorlinks=true,
      urlcolor=urlcolor,
      linkcolor=linkcolor,
      citecolor=citecolor,
      }
    % Slightly bigger margins than the latex defaults
    
    \geometry{verbose,tmargin=1in,bmargin=1in,lmargin=1in,rmargin=1in}
    
    

    \begin{document}
    
    
    \maketitle
    
    

    
    \section{Resolución de Año Nuevo}\label{resoluciuxf3n-de-auxf1o-nuevo}

    ¿Hay más búsquedas en Google de los términos: \texttt{diet} (dieta),
\texttt{gym} (gimnasio) y \texttt{finance} (finanzas) en Enero cuando
todos estamos tratando de renovarnos con el inicio de un nuevo año?

Este proyecto busca verificar
\texttt{Los\ Datos\ de\ Tendencia\ de\ Google} obtenidos de la pagina
\texttt{Google\ Trends} sobre la busqueda de las siguientes palabras:
\texttt{diet}, \texttt{gym} y \texttt{finance}. Y se analizara cómo
estas busquedas varían con el tiempo.

Estos datos pueden ser encontrados
\href{https://trends.google.com/trends/explore?date=all\&q=diet,gym,finance}{aquí}.
También se puede descargar los datos en formato \texttt{.csv}, para que
cada persona pueda importar estos datos en sus propios entornos
virtuales de Python y puedan realizar su propio análisis.

En este proyecto primero se obtendrán los datos, luego se visualizarán,
y finalmente se aprenderá acerca de \texttt{las\ Tendencias} y
\texttt{la\ Estacionalidad} de estos datos. El énfasis se centrará en
una exploración visual.

Entonces la pregunta sigue siendo: ¿Podría haber más búsquedas de estos
palabras en Enero cuando todos estamos tratando de empezar un nuevo año?

    \subsection{Importando los Datos}\label{importando-los-datos}

    \begin{Verbatim}[commandchars=\\\{\}]
{\color{incolor}In [{\color{incolor}182}]:} \PY{c+c1}{\PYZsh{} Primero hay que importar los siguientes paquetes}
          \PY{k+kn}{import} \PY{n+nn}{numpy} \PY{k}{as} \PY{n+nn}{np}
          \PY{k+kn}{import} \PY{n+nn}{pandas} \PY{k}{as} \PY{n+nn}{pd}
          \PY{k+kn}{import} \PY{n+nn}{matplotlib}\PY{n+nn}{.}\PY{n+nn}{pyplot} \PY{k}{as} \PY{n+nn}{plt}
          \PY{k+kn}{import} \PY{n+nn}{seaborn} \PY{k}{as} \PY{n+nn}{sns}
          \PY{o}{\PYZpc{}}\PY{k}{matplotlib} inline
          \PY{n}{sns}\PY{o}{.}\PY{n}{set}\PY{p}{(}\PY{p}{)}
\end{Verbatim}


    \begin{itemize}
\tightlist
\item
  Importando los datos descargardos y verificando las primeras filas:
\end{itemize}

    \begin{Verbatim}[commandchars=\\\{\}]
{\color{incolor}In [{\color{incolor}183}]:} \PY{c+c1}{\PYZsh{} Ya que este archivo viene con un encabezado extra, }
          \PY{c+c1}{\PYZsh{} Usaremos skiprows para no incluirlo}
          \PY{n}{df} \PY{o}{=} \PY{n}{pd}\PY{o}{.}\PY{n}{read\PYZus{}csv}\PY{p}{(}\PY{l+s+s2}{\PYZdq{}}\PY{l+s+s2}{data/multiTimeline.csv}\PY{l+s+s2}{\PYZdq{}}\PY{p}{,} \PY{n}{skiprows} \PY{o}{=} \PY{l+m+mi}{1}\PY{p}{)}
          \PY{n}{df}\PY{o}{.}\PY{n}{head}\PY{p}{(}\PY{p}{)}
\end{Verbatim}


\begin{Verbatim}[commandchars=\\\{\}]
{\color{outcolor}Out[{\color{outcolor}183}]:}      Month  diet: (Worldwide)  gym: (Worldwide)  finance: (Worldwide)
          0  2004-01                100                31                    48
          1  2004-02                 75                26                    49
          2  2004-03                 67                24                    47
          3  2004-04                 70                22                    48
          4  2004-05                 72                22                    43
\end{Verbatim}
            
    \begin{itemize}
\tightlist
\item
  Usando el método \texttt{.info\ ()} para verificar los tipos de datos,
  número de filas y más:
\end{itemize}

    \begin{Verbatim}[commandchars=\\\{\}]
{\color{incolor}In [{\color{incolor}184}]:} \PY{n}{df}\PY{o}{.}\PY{n}{info}\PY{p}{(}\PY{p}{)}
\end{Verbatim}


    \begin{Verbatim}[commandchars=\\\{\}]
<class 'pandas.core.frame.DataFrame'>
RangeIndex: 168 entries, 0 to 167
Data columns (total 4 columns):
Month                   168 non-null object
diet: (Worldwide)       168 non-null int64
gym: (Worldwide)        168 non-null int64
finance: (Worldwide)    168 non-null int64
dtypes: int64(3), object(1)
memory usage: 5.3+ KB

    \end{Verbatim}

    \textbf{Resumen:}

\begin{itemize}
\tightlist
\item
  Se importaron los datos de las tendencias de Google y fueron revisadas
  brevemente;
\end{itemize}

\textbf{A Continuacion:}

\begin{itemize}
\tightlist
\item
  Preparacion de datos y poniendolos en la forma deseada para el
  análisis.
\end{itemize}

    \subsection{Preparacion de Datos}\label{preparacion-de-datos}

    \begin{itemize}
\tightlist
\item
  Cambiamos el nombre de las columnas de \texttt{df} para que no tengan
  espacios:
\end{itemize}

    \begin{Verbatim}[commandchars=\\\{\}]
{\color{incolor}In [{\color{incolor}185}]:} \PY{n}{df}\PY{o}{.}\PY{n}{columns} \PY{o}{=} \PY{p}{[}\PY{l+s+s1}{\PYZsq{}}\PY{l+s+s1}{mes}\PY{l+s+s1}{\PYZsq{}}\PY{p}{,}\PY{l+s+s1}{\PYZsq{}}\PY{l+s+s1}{dieta}\PY{l+s+s1}{\PYZsq{}}\PY{p}{,} \PY{l+s+s1}{\PYZsq{}}\PY{l+s+s1}{gimnasio}\PY{l+s+s1}{\PYZsq{}}\PY{p}{,} \PY{l+s+s1}{\PYZsq{}}\PY{l+s+s1}{finanzas}\PY{l+s+s1}{\PYZsq{}}\PY{p}{]}
          \PY{n}{df}\PY{o}{.}\PY{n}{head}\PY{p}{(}\PY{p}{)}
\end{Verbatim}


\begin{Verbatim}[commandchars=\\\{\}]
{\color{outcolor}Out[{\color{outcolor}185}]:}        mes  dieta  gimnasio  finanzas
          0  2004-01    100        31        48
          1  2004-02     75        26        49
          2  2004-03     67        24        47
          3  2004-04     70        22        48
          4  2004-05     72        22        43
\end{Verbatim}
            
    \begin{itemize}
\item
  Convertimos la columna \texttt{month} en un tipo de
  \texttt{datos\ de\ fecha\ y\ hora} (to\_datetime). Luego hacemos que
  esa columna sea el nuevo índice del DataFrame;
\item
  Hay que tener cuidado cuando usen \texttt{.set\_index()} porque crea
  un nuevo DataFrame en vez de alterar el original. Para poder alterar
  el original tienen que agregarle el parametro \texttt{inplace=True};
\end{itemize}

    \begin{Verbatim}[commandchars=\\\{\}]
{\color{incolor}In [{\color{incolor}186}]:} \PY{n}{df}\PY{o}{.}\PY{n}{mes} \PY{o}{=} \PY{n}{pd}\PY{o}{.}\PY{n}{to\PYZus{}datetime}\PY{p}{(}\PY{n}{df}\PY{o}{.}\PY{n}{mes}\PY{p}{)}
          \PY{n}{df}\PY{o}{.}\PY{n}{set\PYZus{}index}\PY{p}{(}\PY{l+s+s1}{\PYZsq{}}\PY{l+s+s1}{mes}\PY{l+s+s1}{\PYZsq{}}\PY{p}{,} \PY{n}{inplace}\PY{o}{=}\PY{k+kc}{True}\PY{p}{)}\PY{p}{;}
\end{Verbatim}


    \begin{Verbatim}[commandchars=\\\{\}]
{\color{incolor}In [{\color{incolor}187}]:} \PY{n}{df}\PY{o}{.}\PY{n}{head}\PY{p}{(}\PY{p}{)}
\end{Verbatim}


\begin{Verbatim}[commandchars=\\\{\}]
{\color{outcolor}Out[{\color{outcolor}187}]:}             dieta  gimnasio  finanzas
          mes                                  
          2004-01-01    100        31        48
          2004-02-01     75        26        49
          2004-03-01     67        24        47
          2004-04-01     70        22        48
          2004-05-01     72        22        43
\end{Verbatim}
            
    \begin{itemize}
\tightlist
\item
  Ahora es el momento de explorar este DataFrame visualmente.
\end{itemize}

    \subsection{Un Poco de Análisis Exploratorio de
Datos}\label{un-poco-de-anuxe1lisis-exploratorio-de-datos}

    \begin{itemize}
\tightlist
\item
  Usando un método integrado de visualización de \texttt{pandas} para
  trazar los datos como gráficos de 3 líneas en una sola figura (uno
  para cada columna):
\end{itemize}

    \begin{Verbatim}[commandchars=\\\{\}]
{\color{incolor}In [{\color{incolor}188}]:} \PY{c+c1}{\PYZsh{} Vamos a hacer que nuestro grafico se vea mejor}
          \PY{c+c1}{\PYZsh{} Usando los parametros: figsize, linewidth, fontsize }
          \PY{n}{df}\PY{o}{.}\PY{n}{plot}\PY{p}{(}\PY{n}{figsize}\PY{o}{=}\PY{p}{(}\PY{l+m+mi}{20}\PY{p}{,}\PY{l+m+mi}{10}\PY{p}{)}\PY{p}{,} \PY{n}{linewidth}\PY{o}{=}\PY{l+m+mi}{5}\PY{p}{,} \PY{n}{fontsize}\PY{o}{=}\PY{l+m+mi}{20}\PY{p}{)}\PY{p}{;}
          
          \PY{c+c1}{\PYZsh{} Cambiamos la etiqueta del eje\PYZhy{}x a \PYZdq{}Año\PYZdq{}}
          \PY{n}{plt}\PY{o}{.}\PY{n}{xlabel}\PY{p}{(}\PY{l+s+s2}{\PYZdq{}}\PY{l+s+s2}{Año}\PY{l+s+s2}{\PYZdq{}}\PY{p}{,} \PY{n}{fontsize}\PY{o}{=}\PY{l+m+mi}{20}\PY{p}{)}\PY{p}{;}
\end{Verbatim}


    \begin{center}
    \adjustimage{max size={0.9\linewidth}{0.9\paperheight}}{output_18_0.png}
    \end{center}
    { \hspace*{\fill} \\}
    
    \begin{itemize}
\tightlist
\item
  Ahora vamos a trazar solo la columna 'dieta' como una serie temporal:
\end{itemize}

    \begin{Verbatim}[commandchars=\\\{\}]
{\color{incolor}In [{\color{incolor}189}]:} \PY{c+c1}{\PYZsh{} Vamos a hacer que nuestro grafico se vea mejor}
          \PY{c+c1}{\PYZsh{} Usando los parametros: figsize, linewidth, fontsize }
          \PY{n}{df}\PY{p}{[}\PY{p}{[}\PY{l+s+s1}{\PYZsq{}}\PY{l+s+s1}{dieta}\PY{l+s+s1}{\PYZsq{}}\PY{p}{]}\PY{p}{]}\PY{o}{.}\PY{n}{plot}\PY{p}{(}\PY{n}{figsize}\PY{o}{=}\PY{p}{(}\PY{l+m+mi}{20}\PY{p}{,}\PY{l+m+mi}{10}\PY{p}{)}\PY{p}{,} \PY{n}{linewidth}\PY{o}{=}\PY{l+m+mi}{5}\PY{p}{,} \PY{n}{fontsize}\PY{o}{=}\PY{l+m+mi}{20}\PY{p}{)}\PY{p}{;}
          
          \PY{c+c1}{\PYZsh{} Cambiamos la etiqueta del eje\PYZhy{}x a \PYZdq{}Año\PYZdq{}}
          \PY{n}{plt}\PY{o}{.}\PY{n}{xlabel}\PY{p}{(}\PY{l+s+s2}{\PYZdq{}}\PY{l+s+s2}{Año}\PY{l+s+s2}{\PYZdq{}}\PY{p}{,} \PY{n}{fontsize}\PY{o}{=}\PY{l+m+mi}{20}\PY{p}{)}\PY{p}{;}
\end{Verbatim}


    \begin{center}
    \adjustimage{max size={0.9\linewidth}{0.9\paperheight}}{output_20_0.png}
    \end{center}
    { \hspace*{\fill} \\}
    
    ** Nota: ** parece que hay \texttt{tendencias} y componentes
\texttt{estacionales} en estas series temporales. Lo primero que podemos
notar es que hay \texttt{Estacionalidad}, en cada Enero (comienzo de
año) hay un salto alto en la grafica. Pero tambien notamos que hay una
\texttt{Tendencia} que empieza desde arriba, va bajando y luego sube un
poco.

    \textbf{Resumen:}

\begin{itemize}
\tightlist
\item
  Se importaron los datos de las tendencias de Google y fueron revisados
  brevemente;
\item
  Se ha discutido los datos y se convirtieron a la forma deseada para
  prepararlos para el análisis.
\item
  Se ha verificado la serie temporal visualmente.
\end{itemize}

\textbf{A continuacion:}

\begin{itemize}
\tightlist
\item
  Se identificara las tendencias en la serie de tiempo.
\end{itemize}

    Para obtener más información sobre Pandas, pueden ver el curso de la
pagina DataCamp:
\href{https://www.datacamp.com/tracks/data-manipulation-with-python}{Manipulación
de datos usando Python}. Y para obtener más información sobre
manipulacion de series de tiempo con Pandas, pueden ver el curso:
\href{https://www.datacamp.com/courses/manipulating-time-series-data-in-python}{Manipulando
Datos en Series de Tiempo usando Python}.

Si estás disfrutando de esta sesión, visita mi pagina de Facebook
\href{https://www.facebook.com/pythonfordatascience}{Python Programming
\& Data Science} y sígueme en Twitter:
{[}@vmc62usa{]}(https://twitter.com/vmc62usa)

    \subsection{Hay alguna Tendencia?}\label{hay-alguna-tendencia}

    Hay varias maneras de pensar sobre la identificación de tendencias en
series de tiempo. Una manera popular es tomar el
\texttt{Promedio\ Móvil}, es decir, para cada punto de tiempo se toma el
promedio de los puntos de cada extremo (el número de puntos es
especificado por el \emph{tamaño de ventana}, el cual debemos elegir).

    \subsubsection{Visualizando el Promedio
Móvil:}\label{visualizando-el-promedio-muxf3vil}

    \begin{itemize}
\tightlist
\item
  Trazamos el \texttt{Promedio\ Móvil} de \texttt{dieta} utilizando los
  métodos integrados de \texttt{pandas}.
\item
  ¿Qué tamaño de ventana tiene sentido usar?
\end{itemize}

    Tomaremos el Promedio móvil de todo un año porque eso tiende a arreglar
la Estacionalidad. Eso significa que usaremos un
\texttt{tamaño\ de\ ventana} de 12 meses

    \begin{Verbatim}[commandchars=\\\{\}]
{\color{incolor}In [{\color{incolor}190}]:} \PY{c+c1}{\PYZsh{} Creamos un nuevo DataFrame llamado \PYZsq{}dieta\PYZsq{} usando de la columna \PYZsq{}dieta\PYZsq{} de nuestro DataFrame original}
          \PY{n}{dieta} \PY{o}{=} \PY{n}{df}\PY{p}{[}\PY{p}{[}\PY{l+s+s1}{\PYZsq{}}\PY{l+s+s1}{dieta}\PY{l+s+s1}{\PYZsq{}}\PY{p}{]}\PY{p}{]}
          
          \PY{c+c1}{\PYZsh{} Luego tomamos el Promedio Movil de 12 meses y lo trazamos}
          \PY{n}{dieta}\PY{o}{.}\PY{n}{rolling}\PY{p}{(}\PY{l+m+mi}{12}\PY{p}{)}\PY{o}{.}\PY{n}{mean}\PY{p}{(}\PY{p}{)}\PY{o}{.}\PY{n}{plot}\PY{p}{(}\PY{n}{figsize}\PY{o}{=}\PY{p}{(}\PY{l+m+mi}{20}\PY{p}{,}\PY{l+m+mi}{10}\PY{p}{)}\PY{p}{,} \PY{n}{linewidth}\PY{o}{=}\PY{l+m+mi}{5}\PY{p}{,} \PY{n}{fontsize}\PY{o}{=}\PY{l+m+mi}{20}\PY{p}{)}\PY{p}{;}
          
          \PY{c+c1}{\PYZsh{} Cambiamos la etiqueta del eje\PYZhy{}x a \PYZdq{}Año\PYZdq{}}
          \PY{n}{plt}\PY{o}{.}\PY{n}{xlabel}\PY{p}{(}\PY{l+s+s2}{\PYZdq{}}\PY{l+s+s2}{Año}\PY{l+s+s2}{\PYZdq{}}\PY{p}{,} \PY{n}{fontsize}\PY{o}{=}\PY{l+m+mi}{20}\PY{p}{)}\PY{p}{;}
\end{Verbatim}


    \begin{center}
    \adjustimage{max size={0.9\linewidth}{0.9\paperheight}}{output_29_0.png}
    \end{center}
    { \hspace*{\fill} \\}
    
    Removimos la mayoria de la Estacionalidad y podemos ver claramente la
\texttt{Tendencia} de la palabra \texttt{dieta} a traves de los años.

    \begin{itemize}
\tightlist
\item
  Ahora trazamos el Promedio Móvil de \texttt{gimnasio} utilizando los
  métodos integrados de pandas.
\item
  Y de nuevo usaremos un tamaño de ventana de 12 meses
\end{itemize}

    \begin{Verbatim}[commandchars=\\\{\}]
{\color{incolor}In [{\color{incolor}191}]:} \PY{c+c1}{\PYZsh{} Creamos un nuevo DataFrame llamado \PYZsq{}gimnasio\PYZsq{} usando de la columna \PYZsq{}gimnasio\PYZsq{} de nuestro DataFrame original}
          \PY{n}{gimnasio} \PY{o}{=} \PY{n}{df}\PY{p}{[}\PY{p}{[}\PY{l+s+s1}{\PYZsq{}}\PY{l+s+s1}{gimnasio}\PY{l+s+s1}{\PYZsq{}}\PY{p}{]}\PY{p}{]}
          
          \PY{c+c1}{\PYZsh{} Luego tomamos el Promedio Movil de 12 meses y lo trazamos}
          \PY{n}{gimnasio}\PY{o}{.}\PY{n}{rolling}\PY{p}{(}\PY{l+m+mi}{12}\PY{p}{)}\PY{o}{.}\PY{n}{mean}\PY{p}{(}\PY{p}{)}\PY{o}{.}\PY{n}{plot}\PY{p}{(}\PY{n}{figsize}\PY{o}{=}\PY{p}{(}\PY{l+m+mi}{20}\PY{p}{,}\PY{l+m+mi}{10}\PY{p}{)}\PY{p}{,} \PY{n}{linewidth}\PY{o}{=}\PY{l+m+mi}{5}\PY{p}{,} \PY{n}{fontsize}\PY{o}{=}\PY{l+m+mi}{20}\PY{p}{)}\PY{p}{;}
          
          \PY{c+c1}{\PYZsh{} Cambiamos la etiqueta del eje\PYZhy{}x a \PYZdq{}Año\PYZdq{}}
          \PY{n}{plt}\PY{o}{.}\PY{n}{xlabel}\PY{p}{(}\PY{l+s+s2}{\PYZdq{}}\PY{l+s+s2}{Año}\PY{l+s+s2}{\PYZdq{}}\PY{p}{,} \PY{n}{fontsize}\PY{o}{=}\PY{l+m+mi}{20}\PY{p}{)}\PY{p}{;}
\end{Verbatim}


    \begin{center}
    \adjustimage{max size={0.9\linewidth}{0.9\paperheight}}{output_32_0.png}
    \end{center}
    { \hspace*{\fill} \\}
    
    De nuevo, removimos la mayoria de la Estacionalidad y podemos ver
claramente la \texttt{Tendencia} de la palabra \texttt{gimnasio} a
traves de los años. Claramente podemos ver un incremento en la busqueda
de esta palabra.

    \begin{itemize}
\tightlist
\item
  Ahora trazamos las tendencias de \texttt{gimnasio} y \texttt{dieta} en
  una sola grafica:
\end{itemize}

    \begin{Verbatim}[commandchars=\\\{\}]
{\color{incolor}In [{\color{incolor}192}]:} \PY{c+c1}{\PYZsh{} Vamos a concatenar (unir) ambos DataFrames usando el metodo pd.concat}
          \PY{c+c1}{\PYZsh{} Queremos concatenarlos como Columnas asi que usaremos el parametro axis=1}
          
          \PY{n}{df\PYZus{}rm} \PY{o}{=} \PY{n}{pd}\PY{o}{.}\PY{n}{concat}\PY{p}{(}\PY{p}{[}\PY{n}{dieta}\PY{o}{.}\PY{n}{rolling}\PY{p}{(}\PY{l+m+mi}{12}\PY{p}{)}\PY{o}{.}\PY{n}{mean}\PY{p}{(}\PY{p}{)}\PY{p}{,} \PY{n}{gimnasio}\PY{o}{.}\PY{n}{rolling}\PY{p}{(}\PY{l+m+mi}{12}\PY{p}{)}\PY{o}{.}\PY{n}{mean}\PY{p}{(}\PY{p}{)}\PY{p}{]}\PY{p}{,} \PY{n}{axis}\PY{o}{=}\PY{l+m+mi}{1}\PY{p}{)}
          \PY{n}{df\PYZus{}rm}\PY{o}{.}\PY{n}{plot}\PY{p}{(}\PY{n}{figsize}\PY{o}{=}\PY{p}{(}\PY{l+m+mi}{20}\PY{p}{,}\PY{l+m+mi}{10}\PY{p}{)}\PY{p}{,} \PY{n}{linewidth}\PY{o}{=}\PY{l+m+mi}{5}\PY{p}{,} \PY{n}{fontsize}\PY{o}{=}\PY{l+m+mi}{20}\PY{p}{)}\PY{p}{;}
          
          \PY{c+c1}{\PYZsh{} Cambiamos la etiqueta del eje\PYZhy{}x a \PYZdq{}Año\PYZdq{}}
          \PY{n}{plt}\PY{o}{.}\PY{n}{xlabel}\PY{p}{(}\PY{l+s+s2}{\PYZdq{}}\PY{l+s+s2}{Año}\PY{l+s+s2}{\PYZdq{}}\PY{p}{,} \PY{n}{fontsize}\PY{o}{=}\PY{l+m+mi}{20}\PY{p}{)}\PY{p}{;}
\end{Verbatim}


    \begin{center}
    \adjustimage{max size={0.9\linewidth}{0.9\paperheight}}{output_35_0.png}
    \end{center}
    { \hspace*{\fill} \\}
    
    Ahora que los tenemos en el mismo eje, removiendo la
\texttt{Estacionalidad}, podemos ver que atraves de los años la palabra
\texttt{dieta} tiene cierta estacionalidad (sube y baja), mientras que
la busqueda de la palabra \texttt{gimnasio} se va incrementando.

    \begin{itemize}
\tightlist
\item
  Ahora que identificamos las tendencias, vamos a concentrarnos en la
  \texttt{Estacionalidad}
\end{itemize}

    \subsection{Patrones de
Estacionalidad}\label{patrones-de-estacionalidad}

    Se puede eliminar la \texttt{tendencia} de una serie temporal para
investigar su \texttt{estacionalidad}. Es decir, la
\texttt{repetitividad} de nuestra serie de tiempo. En nuestra grafica
original pudimos ver que nuestro trazado \texttt{oscilaba} de arriba a
abajo.

\begin{itemize}
\item
  Una manera de eliminar la tendencia es es restando la tendencia
  calculada anteriormente (su Promedio Móvil) de los datos originales.
  Sin embargo, este nuevo calculo dependerá de la cantidad de puntos de
  datos sobre los que se haya promediado.
\item
  Otra forma de eliminar la tendencia se llama \textbf{diferenciación},
  donde se observa la diferencia entre los puntos de datos sucesivos
  (\texttt{diferenciación\ de\ primer\ orden}).
\end{itemize}

    \subsubsection{Diferenciacion de Primer
Orden}\label{diferenciacion-de-primer-orden}

    \begin{itemize}
\tightlist
\item
  Usamos \texttt{pandas} para calcular y trazar la diferencia de primer
  orden de la serie \texttt{dieta}:
\end{itemize}

    \begin{Verbatim}[commandchars=\\\{\}]
{\color{incolor}In [{\color{incolor}193}]:} \PY{c+c1}{\PYZsh{} Usamos el metodo pd.diff()}
          
          \PY{n}{dieta}\PY{o}{.}\PY{n}{diff}\PY{p}{(}\PY{p}{)}\PY{o}{.}\PY{n}{plot}\PY{p}{(}\PY{n}{figsize}\PY{o}{=}\PY{p}{(}\PY{l+m+mi}{20}\PY{p}{,}\PY{l+m+mi}{10}\PY{p}{)}\PY{p}{,} \PY{n}{linewidth}\PY{o}{=}\PY{l+m+mi}{5}\PY{p}{,} \PY{n}{fontsize}\PY{o}{=}\PY{l+m+mi}{20}\PY{p}{)}\PY{p}{;}
          
          \PY{c+c1}{\PYZsh{} Cambiamos la etiqueta del eje\PYZhy{}x a \PYZdq{}Año\PYZdq{}}
          \PY{n}{plt}\PY{o}{.}\PY{n}{xlabel}\PY{p}{(}\PY{l+s+s2}{\PYZdq{}}\PY{l+s+s2}{Año}\PY{l+s+s2}{\PYZdq{}}\PY{p}{,} \PY{n}{fontsize}\PY{o}{=}\PY{l+m+mi}{20}\PY{p}{)}\PY{p}{;}
\end{Verbatim}


    \begin{center}
    \adjustimage{max size={0.9\linewidth}{0.9\paperheight}}{output_42_0.png}
    \end{center}
    { \hspace*{\fill} \\}
    
    Ahora que removimos la mayor parte de la \texttt{tendencia}, es decir,
podemos ver que la serie esta centrada alrededor de 0 (no se ve la
tendencia que vimos anteriormente). Y aun mas importante es notar que
los \texttt{picos\ en\ la\ grafica} en \texttt{Enero} (comienzos de año)
son mucho mas evidentes ahora. Cada Enero hay una salto tremendo de 20\%
o mas en la busqueda de la palabra \texttt{dieta}.

    \textbf{Nota:} También se puede realizar una
\texttt{diferenciación\ de\ segundo\ orden} si la tendencia aún no se
elimina por completo. Si desean mas informacion sobre
\texttt{diferenciacion} pueden encontrarlo
\href{https://www.otexts.org/fpp/8/1}{aquí}.

\begin{itemize}
\tightlist
\item
  La \textbf{diferenciación} es muy útil para convertir una
  \textbf{serie temporal} en una \textbf{serie temporal estacionaria}.
  Aquí no profundizaremos mucho en esto, pero una
  \texttt{serie\ temporal\ estacionaria} es aquella cuyas propiedades
  estadísticas (como su \textbf{Promedio} y su \textbf{Variancia}) no
  cambian con el tiempo. Las \textbf{series temporales estacionarias}
  son útiles porque muchos métodos de pronóstico de series de tiempo se
  basan en la suposición de que la serie temporal es aproximadamente
  estacionaria.
\end{itemize}

    \textbf{Resumen:}

\begin{itemize}
\tightlist
\item
  Se importaron los datos de las tendencias de Google y fueron revisados
  brevemente;
\item
  Se ha discutido los datos y se convirtieron a la forma deseada para
  prepararlos para el análisis.
\item
  Se ha verificado la serie temporal visualmente.
\item
  Se ha identificado tendencias de la serie temporal.
\item
  Se ha usado la diferenciación de primer orden de series de tiempo.
\end{itemize}

\textbf{A continuacion:}

\begin{itemize}
\tightlist
\item
  Se analizara la periodicidad de la serie de tiempo observando su
  función de autocorrelación;
\item
  Pero antes: un breve desvío para hablar sobre \textbf{correlación.}
\end{itemize}

    \subsubsection{Periodicidad y
Autocorrelación}\label{periodicidad-y-autocorrelaciuxf3n}

    Una serie temporal es \textbf{\emph{periódica}} si se repite a
intervalos igualmente espaciados, por ejemplo, cada 12 meses. Otra forma
de pensar en esto es que si la serie tiene un máximo en algún lugar,
tendrá un máximo de 12 meses después, y si tiene un valle en alguna
parte, también tendrá un valle 12 meses después de eso.

\begin{itemize}
\item
  Otra forma de pensar sobre esto es que la serie de tiempo está
  \textbf{\emph{correlacionada}} con ella misma pero adelantada en 12
  meses.
\item
  El concepto de \textbf{\emph{autocorrelación}} captura la correlación
  de una serie de tiempo con su versión adelantada.
\item
  Primero, recordemos que es \textbf{correlación}:
\end{itemize}

    \subsubsection{Correlación}\label{correlaciuxf3n}

    El coeficiente de correlación de dos variables capta que tan linearmente
relacionadas están:

    \begin{itemize}
\tightlist
\item
  Importamos el conjunto de datos de la flor Iris del paquete
  \texttt{scikit-learn}, luego lo convertimos en un DataFrame, y vemos
  las primeras filas:
\end{itemize}

    

    \begin{Verbatim}[commandchars=\\\{\}]
{\color{incolor}In [{\color{incolor}194}]:} \PY{k+kn}{from} \PY{n+nn}{sklearn} \PY{k}{import} \PY{n}{datasets}
          \PY{n}{iris} \PY{o}{=} \PY{n}{datasets}\PY{o}{.}\PY{n}{load\PYZus{}iris}\PY{p}{(}\PY{p}{)}
          \PY{n}{df\PYZus{}iris} \PY{o}{=} \PY{n}{pd}\PY{o}{.}\PY{n}{DataFrame}\PY{p}{(}\PY{n}{data}\PY{o}{=} \PY{n}{np}\PY{o}{.}\PY{n}{c\PYZus{}}\PY{p}{[}\PY{n}{iris}\PY{p}{[}\PY{l+s+s1}{\PYZsq{}}\PY{l+s+s1}{data}\PY{l+s+s1}{\PYZsq{}}\PY{p}{]}\PY{p}{,} \PY{n}{iris}\PY{p}{[}\PY{l+s+s1}{\PYZsq{}}\PY{l+s+s1}{target}\PY{l+s+s1}{\PYZsq{}}\PY{p}{]}\PY{p}{]}\PY{p}{,}
                               \PY{n}{columns}\PY{o}{=} \PY{n}{iris}\PY{p}{[}\PY{l+s+s1}{\PYZsq{}}\PY{l+s+s1}{feature\PYZus{}names}\PY{l+s+s1}{\PYZsq{}}\PY{p}{]} \PY{o}{+} \PY{p}{[}\PY{l+s+s1}{\PYZsq{}}\PY{l+s+s1}{target}\PY{l+s+s1}{\PYZsq{}}\PY{p}{]}\PY{p}{)}
          \PY{n}{df\PYZus{}iris}\PY{o}{.}\PY{n}{head}\PY{p}{(}\PY{p}{)}
\end{Verbatim}


\begin{Verbatim}[commandchars=\\\{\}]
{\color{outcolor}Out[{\color{outcolor}194}]:}    sepal length (cm)  sepal width (cm)  petal length (cm)  petal width (cm)  \textbackslash{}
          0                5.1               3.5                1.4               0.2   
          1                4.9               3.0                1.4               0.2   
          2                4.7               3.2                1.3               0.2   
          3                4.6               3.1                1.5               0.2   
          4                5.0               3.6                1.4               0.2   
          
             target  
          0     0.0  
          1     0.0  
          2     0.0  
          3     0.0  
          4     0.0  
\end{Verbatim}
            
    \begin{itemize}
\tightlist
\item
  Utilizando \texttt{pandas} o \texttt{seaborn} para construir un
  diagrama de dispersión de \texttt{sepal\ length} (longitud del sépalo
  de la flor) vs. \texttt{sepal\ width} (ancho del sépalo de la flor), y
  coloreado segun el \texttt{target} (objetivo - que es la especie de
  flor):
\end{itemize}

    \begin{Verbatim}[commandchars=\\\{\}]
{\color{incolor}In [{\color{incolor}195}]:} \PY{c+c1}{\PYZsh{} Primero sin color}
          \PY{c+c1}{\PYZsh{} Usando Seaborn funcion de regresion linear \PYZsq{}lmplot\PYZsq{} solo para crear el diagrama de dispersion}
          \PY{c+c1}{\PYZsh{} apagamos la regresion usando el parametro \PYZsq{}fit\PYZus{}reg=False\PYZsq{}}
          
          \PY{n}{sns}\PY{o}{.}\PY{n}{lmplot}\PY{p}{(}\PY{n}{x}\PY{o}{=}\PY{l+s+s1}{\PYZsq{}}\PY{l+s+s1}{sepal length (cm)}\PY{l+s+s1}{\PYZsq{}}\PY{p}{,} \PY{n}{y}\PY{o}{=}\PY{l+s+s1}{\PYZsq{}}\PY{l+s+s1}{sepal width (cm)}\PY{l+s+s1}{\PYZsq{}}\PY{p}{,} \PY{n}{fit\PYZus{}reg}\PY{o}{=}\PY{k+kc}{False}\PY{p}{,} \PY{n}{data}\PY{o}{=}\PY{n}{df\PYZus{}iris}\PY{p}{)}\PY{p}{;}
\end{Verbatim}


    \begin{center}
    \adjustimage{max size={0.9\linewidth}{0.9\paperheight}}{output_54_0.png}
    \end{center}
    { \hspace*{\fill} \\}
    
    \begin{Verbatim}[commandchars=\\\{\}]
{\color{incolor}In [{\color{incolor}196}]:} \PY{c+c1}{\PYZsh{} Con color}
          \PY{n}{sns}\PY{o}{.}\PY{n}{lmplot}\PY{p}{(}\PY{n}{x}\PY{o}{=}\PY{l+s+s1}{\PYZsq{}}\PY{l+s+s1}{sepal length (cm)}\PY{l+s+s1}{\PYZsq{}}\PY{p}{,} \PY{n}{y}\PY{o}{=}\PY{l+s+s1}{\PYZsq{}}\PY{l+s+s1}{sepal width (cm)}\PY{l+s+s1}{\PYZsq{}}\PY{p}{,} \PY{n}{fit\PYZus{}reg}\PY{o}{=}\PY{k+kc}{False}\PY{p}{,} \PY{n}{data}\PY{o}{=}\PY{n}{df\PYZus{}iris}\PY{p}{,} \PY{n}{hue}\PY{o}{=}\PY{l+s+s1}{\PYZsq{}}\PY{l+s+s1}{target}\PY{l+s+s1}{\PYZsq{}}\PY{p}{)}\PY{p}{;}
\end{Verbatim}


    \begin{center}
    \adjustimage{max size={0.9\linewidth}{0.9\paperheight}}{output_55_0.png}
    \end{center}
    { \hspace*{\fill} \\}
    
    \begin{itemize}
\tightlist
\item
  Calcule los coeficientes de correlación de cada par de mediciones:
\end{itemize}

    \begin{Verbatim}[commandchars=\\\{\}]
{\color{incolor}In [{\color{incolor}197}]:} \PY{n}{df\PYZus{}iris}\PY{o}{.}\PY{n}{corr}\PY{p}{(}\PY{p}{)}
\end{Verbatim}


\begin{Verbatim}[commandchars=\\\{\}]
{\color{outcolor}Out[{\color{outcolor}197}]:}                    sepal length (cm)  sepal width (cm)  petal length (cm)  \textbackslash{}
          sepal length (cm)           1.000000         -0.109369           0.871754   
          sepal width (cm)           -0.109369          1.000000          -0.420516   
          petal length (cm)           0.871754         -0.420516           1.000000   
          petal width (cm)            0.817954         -0.356544           0.962757   
          target                      0.782561         -0.419446           0.949043   
          
                             petal width (cm)    target  
          sepal length (cm)          0.817954  0.782561  
          sepal width (cm)          -0.356544 -0.419446  
          petal length (cm)          0.962757  0.949043  
          petal width (cm)           1.000000  0.956464  
          target                     0.956464  1.000000  
\end{Verbatim}
            
    \textbf{Pregunta:}

    \begin{enumerate}
\def\labelenumi{\arabic{enumi})}
\tightlist
\item
  ¿La longitud y el ancho del sépalo se correlacionan positiva o
  negativamente en todas las flores?
\end{enumerate}

    \begin{enumerate}
\def\labelenumi{\arabic{enumi})}
\setcounter{enumi}{1}
\tightlist
\item
  ¿Están correlacionados positiva o negativamente dentro de cada
  especie? Esta es una distinción esencial.
\end{enumerate}

    \begin{itemize}
\tightlist
\item
  Viendo la grafica sin color: Notamos que la \textbf{'longitud del
  sépalo (cm)'} y el \textbf{'ancho del sépalo (cm)'} parecen estar
  \textbf{negativamente correlacionados} con un coeficient de
  \textbf{-0.109369}. Y lo son, sobre \textbf{toda la población de
  flores medidas}.
\end{itemize}

    \begin{itemize}
\tightlist
\item
  \textbf{Pero no dentro de cada especie.} Viendo la grafica a colores
  por ejemplo, la primera especie (color azul) vemos una correlacion
  positiva.
\end{itemize}

    \begin{itemize}
\tightlist
\item
  Para los interesados, esto se conoce como \textbf{la paradoja de
  Simpson} y es esencial cuando se piensa en la inferencia causal.
  Pueden leer más al respecto
  \href{http://ftp.cs.ucla.edu/pub/stat_ser/r414.pdf}{aquí}.
\end{itemize}

    \begin{itemize}
\tightlist
\item
  Calculamos los coeficientes de correlación de cada par de mediciones
  dentro de cada especie:
\end{itemize}

    \begin{Verbatim}[commandchars=\\\{\}]
{\color{incolor}In [{\color{incolor}198}]:} \PY{n}{df\PYZus{}iris}\PY{o}{.}\PY{n}{groupby}\PY{p}{(}\PY{p}{[}\PY{l+s+s1}{\PYZsq{}}\PY{l+s+s1}{target}\PY{l+s+s1}{\PYZsq{}}\PY{p}{]}\PY{p}{)}\PY{o}{.}\PY{n}{corr}\PY{p}{(}\PY{p}{)}
\end{Verbatim}


\begin{Verbatim}[commandchars=\\\{\}]
{\color{outcolor}Out[{\color{outcolor}198}]:}                           petal length (cm)  petal width (cm)  \textbackslash{}
          target                                                          
          0.0    petal length (cm)           1.000000          0.306308   
                 petal width (cm)            0.306308          1.000000   
                 sepal length (cm)           0.263874          0.279092   
                 sepal width (cm)            0.176695          0.279973   
          1.0    petal length (cm)           1.000000          0.786668   
                 petal width (cm)            0.786668          1.000000   
                 sepal length (cm)           0.754049          0.546461   
                 sepal width (cm)            0.560522          0.663999   
          2.0    petal length (cm)           1.000000          0.322108   
                 petal width (cm)            0.322108          1.000000   
                 sepal length (cm)           0.864225          0.281108   
                 sepal width (cm)            0.401045          0.537728   
          
                                    sepal length (cm)  sepal width (cm)  
          target                                                         
          0.0    petal length (cm)           0.263874          0.176695  
                 petal width (cm)            0.279092          0.279973  
                 sepal length (cm)           1.000000          0.746780  
                 sepal width (cm)            0.746780          1.000000  
          1.0    petal length (cm)           0.754049          0.560522  
                 petal width (cm)            0.546461          0.663999  
                 sepal length (cm)           1.000000          0.525911  
                 sepal width (cm)            0.525911          1.000000  
          2.0    petal length (cm)           0.864225          0.401045  
                 petal width (cm)            0.281108          0.537728  
                 sepal length (cm)           1.000000          0.457228  
                 sepal width (cm)            0.457228          1.000000  
\end{Verbatim}
            
    Podemos confirmar que dentro de cada especie las correlaciones son
positivas. Por ejemplo para \texttt{target\ 0.0} (objetivo 0.0) - la
correlacion de \textbf{longitud de sepalo} vs \textbf{ancho de sepalo}
es \textbf{0.746780}. Esta es una de la razones por la cual uno debe
explorar el conjunto de datos los maximo posible.

    \textbf{Resumen:}

\begin{itemize}
\tightlist
\item
  Se importaron los datos de las tendencias de Google y fueron revisados
  brevemente;
\item
  Se ha discutido los datos y se convirtieron a la forma deseada para
  prepararlos para el análisis.
\item
  Se ha verificado la serie temporal visualmente.
\item
  Se ha identificado tendencias de la serie temporal.
\item
  Se ha usado la diferenciación de primer orden de series de tiempo.
\item
  Hemos recordado el concepto de \textbf{correlacion}, como calcularla y
  hemos aprendido sobre \emph{la paradoja de Simpson}
\end{itemize}

\textbf{A continuacion:}

\begin{itemize}
\tightlist
\item
  Se analizara la periodicidad de la serie de tiempo observando su
  función de autocorrelación.
\end{itemize}

    \subsubsection{Correlación de Series de
Tiempo}\label{correlaciuxf3n-de-series-de-tiempo}

    \begin{itemize}
\tightlist
\item
  Vamos a trazar todas las series de tiempo otra vez para recordar cómo
  lucen:
\end{itemize}

    \begin{Verbatim}[commandchars=\\\{\}]
{\color{incolor}In [{\color{incolor}199}]:} \PY{n}{df}\PY{o}{.}\PY{n}{plot}\PY{p}{(}\PY{n}{figsize}\PY{o}{=}\PY{p}{(}\PY{l+m+mi}{20}\PY{p}{,}\PY{l+m+mi}{10}\PY{p}{)}\PY{p}{,} \PY{n}{linewidth}\PY{o}{=}\PY{l+m+mi}{5}\PY{p}{,} \PY{n}{fontsize}\PY{o}{=}\PY{l+m+mi}{20}\PY{p}{)}\PY{p}{;}
          
          \PY{c+c1}{\PYZsh{} Cambiamos la etiqueta del eje\PYZhy{}x a \PYZdq{}Año\PYZdq{}}
          \PY{n}{plt}\PY{o}{.}\PY{n}{rc}\PY{p}{(}\PY{l+s+s1}{\PYZsq{}}\PY{l+s+s1}{legend}\PY{l+s+s1}{\PYZsq{}}\PY{p}{,} \PY{n}{fontsize}\PY{o}{=}\PY{l+m+mi}{20}\PY{p}{)} 
          \PY{n}{plt}\PY{o}{.}\PY{n}{xlabel}\PY{p}{(}\PY{l+s+s2}{\PYZdq{}}\PY{l+s+s2}{Año}\PY{l+s+s2}{\PYZdq{}}\PY{p}{,} \PY{n}{fontsize}\PY{o}{=}\PY{l+m+mi}{20}\PY{p}{)}\PY{p}{;}
\end{Verbatim}


    \begin{center}
    \adjustimage{max size={0.9\linewidth}{0.9\paperheight}}{output_70_0.png}
    \end{center}
    { \hspace*{\fill} \\}
    
    \begin{itemize}
\tightlist
\item
  Calculamos los coeficientes de correlación de todas estas series de
  tiempo:
\end{itemize}

    \begin{Verbatim}[commandchars=\\\{\}]
{\color{incolor}In [{\color{incolor}200}]:} \PY{n}{df}\PY{o}{.}\PY{n}{corr}\PY{p}{(}\PY{p}{)}
\end{Verbatim}


\begin{Verbatim}[commandchars=\\\{\}]
{\color{outcolor}Out[{\color{outcolor}200}]:}              dieta  gimnasio  finanzas
          dieta     1.000000 -0.100764 -0.034639
          gimnasio -0.100764  1.000000 -0.284279
          finanzas -0.034639 -0.284279  1.000000
\end{Verbatim}
            
    \begin{itemize}
\tightlist
\item
  Podemos ver que \textbf{dieta} y \textbf{gimnasio} estan negativamente
  correlacionadas (\textbf{-0.100764}). Es interesante porque, los
  componentes \textbf{Estacionales} estan correlacionados, pero
  \textbf{la tendencia} no lo esta. Hay una correlacion negativa en
  \textbf{la tendencia} de \textbf{dieta} y \textbf{gimnasio}
\end{itemize}

    \begin{itemize}
\tightlist
\item
  Ahora trazaremos las \textbf{diferencias de primer orden} de estas
  series temporales (eliminando \textbf{la tendencia} puede revelar una
  correlación en \textbf{la estacionalidad}):
\end{itemize}

    \begin{Verbatim}[commandchars=\\\{\}]
{\color{incolor}In [{\color{incolor}201}]:} \PY{n}{df}\PY{o}{.}\PY{n}{diff}\PY{p}{(}\PY{p}{)}\PY{o}{.}\PY{n}{plot}\PY{p}{(}\PY{n}{figsize}\PY{o}{=}\PY{p}{(}\PY{l+m+mi}{20}\PY{p}{,}\PY{l+m+mi}{10}\PY{p}{)}\PY{p}{,} \PY{n}{linewidth}\PY{o}{=}\PY{l+m+mi}{5}\PY{p}{,} \PY{n}{fontsize}\PY{o}{=}\PY{l+m+mi}{20}\PY{p}{)}\PY{p}{;}
          
          \PY{c+c1}{\PYZsh{} Cambiamos la etiqueta del eje\PYZhy{}x a \PYZdq{}Año\PYZdq{}}
          \PY{n}{plt}\PY{o}{.}\PY{n}{rc}\PY{p}{(}\PY{l+s+s1}{\PYZsq{}}\PY{l+s+s1}{legend}\PY{l+s+s1}{\PYZsq{}}\PY{p}{,} \PY{n}{fontsize}\PY{o}{=}\PY{l+m+mi}{20}\PY{p}{)} 
          \PY{n}{plt}\PY{o}{.}\PY{n}{xlabel}\PY{p}{(}\PY{l+s+s2}{\PYZdq{}}\PY{l+s+s2}{Año}\PY{l+s+s2}{\PYZdq{}}\PY{p}{,} \PY{n}{fontsize}\PY{o}{=}\PY{l+m+mi}{20}\PY{p}{)}\PY{p}{;}
\end{Verbatim}


    \begin{center}
    \adjustimage{max size={0.9\linewidth}{0.9\paperheight}}{output_75_0.png}
    \end{center}
    { \hspace*{\fill} \\}
    
    Podemos ver que los \textbf{componentes Estacionales} de \texttt{dieta}
y \texttt{gimnasio} estan correlacionados.

    \begin{itemize}
\tightlist
\item
  Ahora calcularemos los \textbf{coeficientes de correlación} de las
  \textbf{diferencias de primer orden} de estas series de tiempo
  (eliminando \textbf{la tendencia} puede revelar una correlación en
  \textbf{la estacionalidad}):
\end{itemize}

    \begin{Verbatim}[commandchars=\\\{\}]
{\color{incolor}In [{\color{incolor}202}]:} \PY{n}{df}\PY{o}{.}\PY{n}{diff}\PY{p}{(}\PY{p}{)}\PY{o}{.}\PY{n}{corr}\PY{p}{(}\PY{p}{)}
\end{Verbatim}


\begin{Verbatim}[commandchars=\\\{\}]
{\color{outcolor}Out[{\color{outcolor}202}]:}              dieta  gimnasio  finanzas
          dieta     1.000000  0.758707  0.373828
          gimnasio  0.758707  1.000000  0.301111
          finanzas  0.373828  0.301111  1.000000
\end{Verbatim}
            
    Habiendo removido la tendencia, y ahora con solamente los
\textbf{componentes de estacionalidad}, los calculos confirman que
\textbf{dieta} y \textbf{gimnasio} estan altamente correlacionados con
un coeficiente de \textbf{0.758707}

    \subsection{Autocorrelación}\label{autocorrelaciuxf3n}

    Ahora que hemos profundizado en la correlación de variables y la
correlación de series de tiempo, es hora de trazar la autocorrelación de
la serie 'dieta': en el eje x tienes el desfase y en el eje y tienes la
correlación entre la la serie de tiempo está consigo misma en ese
retraso. Por ejemplo: si la serie de tiempo original se repite cada dos
días, esperaría ver un aumento en la función de autocorrelación a los 2
días.

    \begin{itemize}
\tightlist
\item
  Trazamos la función de \textbf{autocorrelación} de la serie de tiempo
  \texttt{dieta}:
\end{itemize}

    \begin{Verbatim}[commandchars=\\\{\}]
{\color{incolor}In [{\color{incolor}203}]:} \PY{n}{plt}\PY{o}{.}\PY{n}{figure}\PY{p}{(}\PY{n}{figsize}\PY{o}{=}\PY{p}{(}\PY{l+m+mi}{20}\PY{p}{,}\PY{l+m+mi}{10}\PY{p}{)}\PY{p}{)}\PY{p}{;}
          \PY{n}{plt}\PY{o}{.}\PY{n}{rc}\PY{p}{(}\PY{l+s+s1}{\PYZsq{}}\PY{l+s+s1}{axes}\PY{l+s+s1}{\PYZsq{}}\PY{p}{,} \PY{n}{labelsize}\PY{o}{=}\PY{l+m+mi}{20}\PY{p}{)} 
          \PY{n}{plt}\PY{o}{.}\PY{n}{rc}\PY{p}{(}\PY{l+s+s1}{\PYZsq{}}\PY{l+s+s1}{xtick}\PY{l+s+s1}{\PYZsq{}}\PY{p}{,} \PY{n}{labelsize}\PY{o}{=}\PY{l+m+mi}{20}\PY{p}{)}
          \PY{n}{plt}\PY{o}{.}\PY{n}{rc}\PY{p}{(}\PY{l+s+s1}{\PYZsq{}}\PY{l+s+s1}{ytick}\PY{l+s+s1}{\PYZsq{}}\PY{p}{,} \PY{n}{labelsize}\PY{o}{=}\PY{l+m+mi}{20}\PY{p}{)}
          \PY{n}{pd}\PY{o}{.}\PY{n}{plotting}\PY{o}{.}\PY{n}{autocorrelation\PYZus{}plot}\PY{p}{(}\PY{n}{dieta}\PY{p}{,} \PY{n}{color}\PY{o}{=}\PY{l+s+s1}{\PYZsq{}}\PY{l+s+s1}{green}\PY{l+s+s1}{\PYZsq{}}\PY{p}{)}\PY{p}{;}
\end{Verbatim}


    \begin{center}
    \adjustimage{max size={0.9\linewidth}{0.9\paperheight}}{output_83_0.png}
    \end{center}
    { \hspace*{\fill} \\}
    
    De nuevo, hemos identificado la \texttt{Estacionalidad}
\textbf{(repetitividad)} de la busqueda de la palabra \texttt{dieta} en
Google cada 12 meses. Por ejemplo podemos ver maximos (picos) cada 12
meses, es decir podemos ver esos maximos en un \texttt{lag}
\textbf{(desfase)} de 24, 36, 48 , y asi sucesivamente.

    \textbf{Resumen:}

\begin{itemize}
\tightlist
\item
  Se importaron los datos de las tendencias de Google y fueron revisados
  brevemente;
\item
  Se ha discutido los datos y se convirtieron a la forma deseada para
  prepararlos para el análisis.
\item
  Se ha verificado la serie temporal visualmente.
\item
  Se ha identificado tendencias de la serie temporal.
\item
  Se ha usado la diferenciación de primer orden de series de tiempo.
\item
  Hemos recordado el concepto de \textbf{correlacion}, como calcularla y
  hemos aprendido sobre \emph{la paradoja de Simpson}
\item
  Se ha analizado la periodicidad de la serie de tiempos observando su
  función de autocorrelación.
\end{itemize}

    En este código de Facebook en vivo a lo largo de la sesión, ha
consultado los datos de las tendencias de Google de las palabras clave
"dieta", "gimnasio" y miró de forma superficial "finanzas" para ver cómo
varían con el tiempo. Para aquellos ansiosos científicos de datos, hay
dos cosas que puedes hacer de inmediato:

\begin{itemize}
\tightlist
\item
  Observe y analice los datos columna 'finanzas' y luego informe lo que
  ha encontrado;
\item
  Use el modelo ARIMA para hacer algunas predicciones de series
  temporales sobre cómo serán estas tendencias de búsqueda en los
  próximos años. Jason Brownlee en Machine Learning Mastery tiene un
  estupendo tutorial sobre
  \href{https://machinelearningmastery.com/arima-for-time-series-forecasting-with-python/}{modelo
  ARIMA en Python}, y ademas \textbf{DataCamp} tiene un
  \href{https://www.datacamp.com/courses/arima-modeling-with-r}{gran
  curso sobre el modelo ARIMA usando R} y \textbf{DataCamp} también
  tendra un curso de previsión Python Time Series funcionando este año.
\end{itemize}

    Para obtener más información sobre Pandas, pueden ver el curso de la
pagina DataCamp:
\href{https://www.datacamp.com/tracks/data-manipulation-with-python}{Manipulación
de datos usando Python}. Y para obtener más información sobre
manipulacion de series de tiempo con Pandas, pueden ver el curso:
\href{https://www.datacamp.com/courses/manipulating-time-series-data-in-python}{Manipulando
Datos en Series de Tiempo usando Python}.

Si estás disfrutando de esta sesión, visita mi pagina de Facebook
\href{https://www.facebook.com/pythonfordatascience}{Python Programming
\& Data Science} y sígueme en Twitter:
{[}@vmc62usa{]}(https://twitter.com/vmc62usa)


    % Add a bibliography block to the postdoc
    
    
    
    \end{document}
